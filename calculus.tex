\documentclass[12pt]{article}
\usepackage{graphicx}

\addtolength{\oddsidemargin}{-0.8in}
\addtolength{\evensidemargin}{-1in}
\addtolength{\textwidth}{1.6in}

\addtolength{\topmargin}{-1in}
\addtolength{\textheight}{1in}

\linespread{1.2}

\begin{document}

  \begin{center}
    \begin{figure}[t]
      \centering
      \includegraphics[width=0.1\textwidth]{cu_logo.png}
    \end{figure}

    {\bfseries \LARGE University of Chittagong\\
    Department of Mathematics\\
    First year B.Sc. (Honours) Examination - 2022\\
    Course Title: Calculus-1\\
    Course Code: Math-102\\[15pt]}

    {\bfseries \large Time: 4 Hours \hspace{16em} Full Marks: 75\\[15pt]}

    {\large \textbf{[Instruction:} Answer any \textbf{05 (Five)} questions. The questions are of equal marks and figures in the margin indicate full marks. \textbf{Answer the several parts of a question sequentially.]}\\[30pt]}
  \end{center}

  \begin{itemize}
    {\large
    \item[Q1.] 
      \begin{itemize}
        {\large
        \item[a)] Discuss the relationship between relation and function. Define One to One function, Identity function and Even function with examples.
        \item[b)] If $f:R-\left\{ \frac{1}{2} \right\}$
        \item[c)] 
        \item[d)]}
      \end{itemize}
    \item[Q2.] 
    \begin{itemize}
      {\large
      \item[a)] 
      \item[b)] 
      \item[c)] 
      \item[d)]}
    \end{itemize}
    \item[Q3.] 
    \begin{itemize}
      {\large
      \item[a)] 
      \item[b)] 
      \item[c)] 
      \item[d)]}
    \end{itemize}
    \item[Q4.] \begin{itemize}
      {\large
      \item[a)] 
      \item[b)] 
      \item[c)] 
      \item[d)]}
    \end{itemize}
    \item[Q5.] \begin{itemize}
      {\large
      \item[a)] 
      \item[b)] 
      \item[c)] 
      \item[d)]}
    \end{itemize}
    \item[Q6.] \begin{itemize}
      {\large
      \item[a)] 
      \item[b)] 
      \item[c)] 
      \item[d)]}
    \end{itemize}
    \item[Q7.] \begin{itemize}
      {\large
      \item[a)] 
      \item[b)] 
      \item[c)] 
      \item[d)]}
    \end{itemize}
    \item[Q8.] \begin{itemize}
      {\large
      \item[a)] 
      \item[b)] 
      \item[c)] 
      \item[d)]}
    \end{itemize}}
  \end{itemize}
\end{document}