\documentclass[12pt]{article}
\usepackage{graphicx, amsmath, amsfonts, amssymb}

\addtolength{\oddsidemargin}{-0.8in}
\addtolength{\evensidemargin}{-1in}
\addtolength{\textwidth}{1.6in}

\addtolength{\topmargin}{-1in}
\addtolength{\textheight}{1in}

\linespread{1.5}
\pagestyle{empty}

\begin{document}

  \begin{center}
    \begin{figure}[t]
      \centering
      \includegraphics[width=0.1\textwidth]{cu_logo.png}
    \end{figure}

    {\bfseries \large University of Chittagong\\
    Department of Mathematics\\
    First year B.Sc. (Honours) Examination - 2022\\
    Course Title: Calculus-1\\
    Course Code: Math-102\\[15pt]}

    {\bfseries \normalsize Time: 4 Hours \hspace{23em} Full Marks: 75\\[15pt]}

    {\normalsize \textbf{[Instruction:} Answer any \textbf{05 (Five)} questions. The questions are of equal marks and figures in the margin indicate full marks. \textbf{Answer the several parts of a question sequentially.]}\\[15pt]}
  \end{center}

  \begin{itemize}
    {\large \linespread{2}
    \item[Q1.] 
      \begin{itemize}
        {\normalsize
        \item[a)] Discuss the relationship between relation and function. Define One to One function, Identity function and Even function with examples.
        \item[b)] If $f:R-\left\{ \dfrac{5}{4} \right\} \to R-\left\{ \dfrac{1}{2} \right\}$ is defined by the formula $f(x)=\dfrac{2x+3}{4x-5}$ then find $y=f^{-1}(x)$.
        \item[c)] A function is given-
        $$f(x) = 
          \begin{cases}
            x^2 & \text{when } x < 0 \\
            x & \text{when } 0 \leq x \leq 1 \\
            \frac{1}{x} & \text{when } x \geq 1
          \end{cases}$$
          \begin{itemize}
            {\large
            \item[i)] Draw the graph of the given function $f(x)$.
            \item[ii)] FInd the Domain and range of $f(x)$.
            \item[iii)] Describe the properties of the graph of $f(x)$.}
          \end{itemize}}
      \end{itemize}
    \item[Q2.] 
    \begin{itemize}
      {\large
      \item[a)] Define limit of function using $(\epsilon - \delta)$. Write the difference between $\lim \limits_{x \to a} f(x)$ and $f(a)$.
      \item[b)] Show that the function 
          $f(x) = 
          \begin{cases}
            x+\dfrac{1}{3} & \text{when } x \neq 0 \\
            0 & \text{when } x = 1
          \end{cases}$ continuous but $f'(x)$ does not exist at $x=0$.
      \item[c)] Using the fundamental theorem of differentiability find differential coefficient of $\tan ax$.}
    \end{itemize}
    \item[Q3.] 
    \begin{itemize}
      {\large
      \item[a)] Find $\dfrac{dy}{dx}$, if (i) $y=(\tan x)^{(\cot x)}$\\(ii) $x^2y+xy^2+\sqrt{xy}=1$
      \item[b)] State and prove Leibnitz theorem for the nth derivative of the product of two functions.
      \item[c)] If $x=\sin \left(\dfrac{1}{m}\log y\right)$, show that\\
      $(a-x^2)y_{n+2}-(2n+1)xy_{n+1}-(n^2+m^2)y_n=0$}
    \end{itemize}
    \item[Q4.] \begin{itemize}
      {\large
      \item[a)] State and prove the Mean value theorem and also justify this theorem for the function $f(x)=3+2x-x^2$ in the interval $(0,1)$
      \item[b)] Show that (Any Two)-\\
        \begin{itemize}
        {\large
        \item[i)] $\lim \limits_{x \to 0} \dfrac{x^2-\sin x^2}{x^6}=\dfrac{1}{6}$
        \item[ii)] $\lim \limits_{\theta \to \pi/4} \dfrac{\sqrt{2}-\cos \theta - \sin \theta}{(4\theta - \pi)^2}=\dfrac{1}{16\sqrt{2}}$
        \item[iii)] $\lim \limits_{x \to 1} \left(\dfrac{x}{x-1}-\dfrac{1}{\ln x}\right)=\dfrac{1}{2}$}
        \end{itemize}
      \item[c)] Find differential coefficient of $\tan^{-1} \left( \dfrac{x}{\sqrt{1-x^2}}\right)$ with respect to\\ $\sec^{-1} \left(\dfrac{1}{2x^2-11}\right)$}
    \end{itemize}
    \item[Q5.] \begin{itemize}
      {\large
      \item[a)] If u be a homogenous function of degree n in x and y then prove that\\
      $x^2\dfrac{\partial^2u}{\partial x^2}+2xy\dfrac{\partial^2u}{\partial y\partial x}+y^2\dfrac{\partial^2u}{\partial y^2}=n(n-1)u$.
      \item[b)] If $u$ be a homigenous function of degree $n$ in $x, y$ and $z$ then verify the relation\\
      $x\dfrac{\partial u}{\partial y}+y\dfrac{\partial u}{\partial y}+z\dfrac{\partial u}{\partial z}=nu$
      \item[c)] If $z=\tan^{-1}\left(\dfrac{x^3+y^3}{x-y}\right)$, then prove that $x\dfrac{\partial z}{\partial x}+y\dfrac{\partial z}{\partial y}=\sin 2z$.}
    \end{itemize}
    \item[Q6.] \begin{itemize}
      {\large
      \item[a)] Integrate of the following (Any two)\\
      \begin{tabular}{ll}
          i) $\int \dfrac{dx}{\sqrt{1-x^2}\sqrt{\sin^{-1}x}}$ & i) $\int \dfrac{\cos x}{a^2+b^2 \sin^2x}\,dx$\\
          iii) $\int x^2\cos x\,dx$ & iv) $\int \dfrac{dx}{5+4\cos x}$
      \end{tabular}
      \item[b)]Find the reduction formula of $\int \cos^nx\,dx$. Hence show that\\$\int_0^\frac{\pi}{2}\cos^7x\,dx=\dfrac{16}{35}$}
    \end{itemize}
    \item[Q7.] \begin{itemize}
      {\large
      \item[a)] Discuss the geometrical interpretation of definite integral.
      \item[b)] Integrate of the following (Any two)\\
        \begin{tabular}{ll}
          i) $\int_0^1\dfrac{\ln x}{\sqrt{1-x^2}}\,dx$ & ii) $\int_2^3\dfrac{x^2+1}{(2x+1)(x^2-1)}\,dx$\\
          iii) $\int_0^\frac{\pi}{4}\ln\sin 2\theta\,d\theta$
        \end{tabular}
      \item[c)] Find the length of an arc of the curve $y=\log \left(\dfrac{e^x-1}{e^x+1}\right)$ from $x=1$ to $x=2$.}
    \end{itemize}
    \item[Q8.] \begin{itemize}
      {\large
      \item[a)] Define Gamma and Beta functions. Show that, $\Gamma(n+1)=n\Gamma n$.
      \item[b)] Show that $\beta(m,n)=\dfrac{\Gamma m\Gamma n}{\Gamma(m+n)}$
      \item[c)] Evaluate: $\int \limits_0^{\frac{\pi}{6}}\cos^43\theta\sin^26\theta\,d\theta$}
    \end{itemize}}
  \end{itemize}
\end{document}
